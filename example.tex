\PassOptionsToPackage{unicode}{hyperref}
\documentclass[t]{beamer}

% option is one of southday, southnight and northday
\usetheme[southnight]{ctaposter}

\usepackage[
  orientation=portrait,
  size=a0,
  scale=1.0,  % scale fonts with this factor
]{beamerposter}

\usepackage{fontspec}
\usepackage[american]{babel}

% microtypographic adjusments (protrusion, kerning etc.)
\usepackage{microtype}

% automatic enquoting depending on the language with \enquote{Bla}
\usepackage{csquotes}

% for the qrcode
\usepackage{qrcode}

% math setup
\usepackage{mathtools}
\usepackage[
  math-style=ISO,
  bold-style=ISO,
  sans-style=italic,
  nabla=upright,
  partial=upright,
]{unicode-math}

% some stupid sample text to fill the boxes
\usepackage{blindtext}

% offers the \begin{multicols}{NCOLS} environment
\usepackage{multicol}

% for including all types of graphics
\usepackage{graphicx}

% bibliography settings
\usepackage[
  backend=biber,
  style=numeric,
  url=false,
  sorting=none,
  firstinits=true,
]{biblatex}
\DeclareFieldFormat*{title}{\textit{#1}}

\hypersetup{colorlinks=true, linkcolor=lightblue, urlcolor=lightblue}


% just a helpful definition to get columns with a third of the textwidth
\newlength{\thirdtextwidth}
\setlength\thirdtextwidth{0.333333333\textwidth}


\title{Title of the Contribution}
\author{%
  M. Peresano\textsuperscript{1},
  K. Kosack\textsuperscript{1},
  and M. Nöthe\textsuperscript{2},
  for the CTA Consortium
}
\institute{%
  \textsuperscript{1} IRFU, CEA Saclay \\
  \textsuperscript{2} Department of Physics, TU Dortmund University \\
}
\abstract{%
  The Cherenkov Telescope Array (CTA) is the next-generation gamma-ray observatory
  currently under construction.
  It will improve over the current generation of imaging atmospheric Cherenkov telescopes (IACTs)
  by at least one order of magnitude in sensitivity and be able to observe the whole
  sky from a northern site in La Palma, Spain, and a southern one in Paranal, Chile.
  CTA will also be the first open gamma-ray observatory.
  Accordingly, the data analysis pipeline is developed as open-source software.
  The event reconstruction pipeline accepts raw data of the telescopes and processes it to
  produce suitable input for the higher-level science tools.
  Its primary tasks include estimating the physical properties of each recorded
  shower and providing the corresponding instrument response functions.
  Ctapipe is a framework providing algorithms and tools to facilitate raw data calibration,
  image extraction, image parameterization and event reconstruction.
  Its main focus is currently the analysis of simulated data but it has also been successfully applied
  for the analysis of data obtained with the first CTA prototype telescopes, such as the Large Size Telescope 1.
  PyIRF is a library to calculate IACT instrument response functions,
  needed to obtain physics results like spectra and light curves,
  from the reconstructed event lists.
  Building on these two, protopipe is a prototype for the event reconstruction pipeline for CTA.
}

\supportedby{%
  \begin{columns}[c, onlytextwidth]
    Put your logos here
  \end{columns}
}

\begin{document}%
\begin{columns}[onlytextwidth]%
  \begin{column}{\textwidth}%
    \begin{block}
      \blindtext[2]
    \end{block}
  \end{column}%
\end{columns}%
\begin{columns}[onlytextwidth]%
  \begin{column}{0.5\textwidth}%
    \begin{block}[equal height group=A]{Test}%
      \begin{multicols}{2}
        \begin{figure}
          \includegraphics[width=\linewidth]{example-image-a}\\
          \caption{Toller Plot.\cite{fact-performance}}\label{fig:tollerplot}
        \end{figure}
        \columnbreak
        \url{https://github.com/cta-observatory/ctapipe}
        \blindtext
      \end{multicols}
    \end{block}%
  \end{column}%
  \begin{column}{0.5\textwidth}%
    \begin{alertblock}[equal height group=A]{Test}%
      \begin{multicols}{2}
        \blindtext\cite{fact-reference}
        \blindtext
      \end{multicols}
    \end{alertblock}%
  \end{column}%
\end{columns}%
\begin{columns}[c, onlytextwidth]%
  \begin{column}{\thirdtextwidth}%
    \begin{exampleblock}{Test}%
      \blindtext%
    \end{exampleblock}%
  \end{column}%
  \begin{column}{\thirdtextwidth}%
    \begin{block}{Test}%
      \blindtext%
    \end{block}%
  \end{column}%
  \begin{column}{\thirdtextwidth}%
    \begin{block}{Test}%
      \blindtext%
    \end{block}%
  \end{column}%
\end{columns}%
\begin{columns}[onlytextwidth]%
  \begin{column}{0.5\textwidth}%
    \begin{block}{Test}%
      \blindtext%
    \end{block}%
  \end{column}%
  \begin{column}{0.5\textwidth}%
    \begin{block}{Test}%
      \blindtext%
    \end{block}%
  \end{column}%
\end{columns}%


\vfill
\begin{columns}[t, onlytextwidth]%
  \begin{column}{0.42\textwidth}%
    \begin{block}[equal height group=bottom]{\normalsize References}
      \footnotesize%
      \printbibliography%
    \end{block}
  \end{column}%
  \begin{column}{0.42\textwidth}%
    \begin{block}[equal height group=bottom]{\normalsize Acknowledgment}
      \begin{multicols}{2}%
        \footnotesize
        We thank all our funding agencies!
      \end{multicols}%
    \end{block}
  \end{column}%
  \begin{column}{0.16\textwidth}%
    \footnotesize
    \begin{block}[equal height group=bottom]{\normalsize Author Information}
    \end{block}
  \end{column}%
\end{columns}%
\end{document}
